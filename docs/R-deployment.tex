\documentclass[]{book}
\usepackage{lmodern}
\usepackage{amssymb,amsmath}
\usepackage{ifxetex,ifluatex}
\usepackage{fixltx2e} % provides \textsubscript
\ifnum 0\ifxetex 1\fi\ifluatex 1\fi=0 % if pdftex
  \usepackage[T1]{fontenc}
  \usepackage[utf8]{inputenc}
\else % if luatex or xelatex
  \ifxetex
    \usepackage{mathspec}
  \else
    \usepackage{fontspec}
  \fi
  \defaultfontfeatures{Ligatures=TeX,Scale=MatchLowercase}
\fi
% use upquote if available, for straight quotes in verbatim environments
\IfFileExists{upquote.sty}{\usepackage{upquote}}{}
% use microtype if available
\IfFileExists{microtype.sty}{%
\usepackage{microtype}
\UseMicrotypeSet[protrusion]{basicmath} % disable protrusion for tt fonts
}{}
\usepackage[margin=1in]{geometry}
\usepackage{hyperref}
\hypersetup{unicode=true,
            pdftitle={R语言模型部署实战},
            pdfauthor={徐静},
            pdfborder={0 0 0},
            breaklinks=true}
\urlstyle{same}  % don't use monospace font for urls
\usepackage{natbib}
\bibliographystyle{apalike}
\usepackage{longtable,booktabs}
\usepackage{graphicx,grffile}
\makeatletter
\def\maxwidth{\ifdim\Gin@nat@width>\linewidth\linewidth\else\Gin@nat@width\fi}
\def\maxheight{\ifdim\Gin@nat@height>\textheight\textheight\else\Gin@nat@height\fi}
\makeatother
% Scale images if necessary, so that they will not overflow the page
% margins by default, and it is still possible to overwrite the defaults
% using explicit options in \includegraphics[width, height, ...]{}
\setkeys{Gin}{width=\maxwidth,height=\maxheight,keepaspectratio}
\IfFileExists{parskip.sty}{%
\usepackage{parskip}
}{% else
\setlength{\parindent}{0pt}
\setlength{\parskip}{6pt plus 2pt minus 1pt}
}
\setlength{\emergencystretch}{3em}  % prevent overfull lines
\providecommand{\tightlist}{%
  \setlength{\itemsep}{0pt}\setlength{\parskip}{0pt}}
\setcounter{secnumdepth}{5}
% Redefines (sub)paragraphs to behave more like sections
\ifx\paragraph\undefined\else
\let\oldparagraph\paragraph
\renewcommand{\paragraph}[1]{\oldparagraph{#1}\mbox{}}
\fi
\ifx\subparagraph\undefined\else
\let\oldsubparagraph\subparagraph
\renewcommand{\subparagraph}[1]{\oldsubparagraph{#1}\mbox{}}
\fi

%%% Use protect on footnotes to avoid problems with footnotes in titles
\let\rmarkdownfootnote\footnote%
\def\footnote{\protect\rmarkdownfootnote}

%%% Change title format to be more compact
\usepackage{titling}

% Create subtitle command for use in maketitle
\newcommand{\subtitle}[1]{
  \posttitle{
    \begin{center}\large#1\end{center}
    }
}

\setlength{\droptitle}{-2em}

  \title{R语言模型部署实战}
    \pretitle{\vspace{\droptitle}\centering\huge}
  \posttitle{\par}
    \author{徐静}
    \preauthor{\centering\large\emph}
  \postauthor{\par}
      \predate{\centering\large\emph}
  \postdate{\par}
    \date{2018-08-06}

\usepackage{booktabs}
\usepackage{xeCJK}

\setCJKmainfont{宋体}

\setmainfont{Georgia}

\setromanfont{Georgia}

\setmonofont{Courier New}

\begin{document}
\maketitle

{
\setcounter{tocdepth}{1}
\tableofcontents
}
\chapter*{序言}
\addcontentsline{toc}{chapter}{序言}

我们的模型不能只停留在线下的分析报告中,训练好的R模型如何应用到生产环境?目前针对于R语言的模型生产环境应用的方式有很多,比如用其他语言去掉用,Java,Python等语言均可方便的调用R脚本;生成PMML文件,目前R中主流的一些R模型均支持PMML比如xgboost,lightGBM等,其他语言不需要调用R脚本只需调用统一的PMML文件就可以;还有就是Web端的部署,比如可以做成REST
API供其他语言调用,或直接做成web应用供其他用户访问,本书主要针对于R语言模型的Web端的部署。过程中,我们会先后介绍httpuv,opencpu,plumber,
jug,fiery,Rserve,RestRserve,等一些和模型线上化部署相关的R包,最后会介绍mailR和Rweixin两个R和邮件与微信通信的R包,用于线上化部署的监测。当然会有其他的线上化部署方式。

欢迎进入R模型线上化部署的海洋!

\chapter*{关于我}
\addcontentsline{toc}{chapter}{关于我}

\textbf{徐静:}

硕士研究生,
目前的研究兴趣主要包括:数理统计,统计机器学习,深度学习,网络爬虫,前端可视化,R语言和Python语言的超级粉丝,多个R包和Python模块的作者,现在正逐步向Java迁移。

Graduate students,the current research interests include: mathematical
statistics, statistical machine learning, deep learning, web crawler,
front-end visualization. He is a super fan of R and Python, and the
author of several R packages and Python modules, and now gradually
migrating to Java.

\chapter{httpuv}\label{httpuv}

\chapter{opencpu}\label{opencpu}

\chapter{plumber}\label{plumber}

\chapter{jug}\label{jug}

\chapter{fiery}\label{fiery}

\chapter{Rserve}\label{rserve}

\chapter{RestRserve}\label{restrserve}

\chapter{mailR}\label{mailr}

\chapter{Rweixin}\label{rweixin}

\chapter{参考文献}\label{reference}

\bibliography{book.bib,packages.bib}


\end{document}
