\documentclass[]{book}
\usepackage{lmodern}
\usepackage{amssymb,amsmath}
\usepackage{ifxetex,ifluatex}
\usepackage{fixltx2e} % provides \textsubscript
\ifnum 0\ifxetex 1\fi\ifluatex 1\fi=0 % if pdftex
  \usepackage[T1]{fontenc}
  \usepackage[utf8]{inputenc}
\else % if luatex or xelatex
  \ifxetex
    \usepackage{mathspec}
  \else
    \usepackage{fontspec}
  \fi
  \defaultfontfeatures{Ligatures=TeX,Scale=MatchLowercase}
\fi
% use upquote if available, for straight quotes in verbatim environments
\IfFileExists{upquote.sty}{\usepackage{upquote}}{}
% use microtype if available
\IfFileExists{microtype.sty}{%
\usepackage{microtype}
\UseMicrotypeSet[protrusion]{basicmath} % disable protrusion for tt fonts
}{}
\usepackage[margin=1in]{geometry}
\usepackage{hyperref}
\hypersetup{unicode=true,
            pdftitle={R语言模型部署实战},
            pdfauthor={徐静},
            pdfborder={0 0 0},
            breaklinks=true}
\urlstyle{same}  % don't use monospace font for urls
\usepackage{natbib}
\bibliographystyle{apalike}
\usepackage{color}
\usepackage{fancyvrb}
\newcommand{\VerbBar}{|}
\newcommand{\VERB}{\Verb[commandchars=\\\{\}]}
\DefineVerbatimEnvironment{Highlighting}{Verbatim}{commandchars=\\\{\}}
% Add ',fontsize=\small' for more characters per line
\usepackage{framed}
\definecolor{shadecolor}{RGB}{248,248,248}
\newenvironment{Shaded}{\begin{snugshade}}{\end{snugshade}}
\newcommand{\KeywordTok}[1]{\textcolor[rgb]{0.13,0.29,0.53}{\textbf{#1}}}
\newcommand{\DataTypeTok}[1]{\textcolor[rgb]{0.13,0.29,0.53}{#1}}
\newcommand{\DecValTok}[1]{\textcolor[rgb]{0.00,0.00,0.81}{#1}}
\newcommand{\BaseNTok}[1]{\textcolor[rgb]{0.00,0.00,0.81}{#1}}
\newcommand{\FloatTok}[1]{\textcolor[rgb]{0.00,0.00,0.81}{#1}}
\newcommand{\ConstantTok}[1]{\textcolor[rgb]{0.00,0.00,0.00}{#1}}
\newcommand{\CharTok}[1]{\textcolor[rgb]{0.31,0.60,0.02}{#1}}
\newcommand{\SpecialCharTok}[1]{\textcolor[rgb]{0.00,0.00,0.00}{#1}}
\newcommand{\StringTok}[1]{\textcolor[rgb]{0.31,0.60,0.02}{#1}}
\newcommand{\VerbatimStringTok}[1]{\textcolor[rgb]{0.31,0.60,0.02}{#1}}
\newcommand{\SpecialStringTok}[1]{\textcolor[rgb]{0.31,0.60,0.02}{#1}}
\newcommand{\ImportTok}[1]{#1}
\newcommand{\CommentTok}[1]{\textcolor[rgb]{0.56,0.35,0.01}{\textit{#1}}}
\newcommand{\DocumentationTok}[1]{\textcolor[rgb]{0.56,0.35,0.01}{\textbf{\textit{#1}}}}
\newcommand{\AnnotationTok}[1]{\textcolor[rgb]{0.56,0.35,0.01}{\textbf{\textit{#1}}}}
\newcommand{\CommentVarTok}[1]{\textcolor[rgb]{0.56,0.35,0.01}{\textbf{\textit{#1}}}}
\newcommand{\OtherTok}[1]{\textcolor[rgb]{0.56,0.35,0.01}{#1}}
\newcommand{\FunctionTok}[1]{\textcolor[rgb]{0.00,0.00,0.00}{#1}}
\newcommand{\VariableTok}[1]{\textcolor[rgb]{0.00,0.00,0.00}{#1}}
\newcommand{\ControlFlowTok}[1]{\textcolor[rgb]{0.13,0.29,0.53}{\textbf{#1}}}
\newcommand{\OperatorTok}[1]{\textcolor[rgb]{0.81,0.36,0.00}{\textbf{#1}}}
\newcommand{\BuiltInTok}[1]{#1}
\newcommand{\ExtensionTok}[1]{#1}
\newcommand{\PreprocessorTok}[1]{\textcolor[rgb]{0.56,0.35,0.01}{\textit{#1}}}
\newcommand{\AttributeTok}[1]{\textcolor[rgb]{0.77,0.63,0.00}{#1}}
\newcommand{\RegionMarkerTok}[1]{#1}
\newcommand{\InformationTok}[1]{\textcolor[rgb]{0.56,0.35,0.01}{\textbf{\textit{#1}}}}
\newcommand{\WarningTok}[1]{\textcolor[rgb]{0.56,0.35,0.01}{\textbf{\textit{#1}}}}
\newcommand{\AlertTok}[1]{\textcolor[rgb]{0.94,0.16,0.16}{#1}}
\newcommand{\ErrorTok}[1]{\textcolor[rgb]{0.64,0.00,0.00}{\textbf{#1}}}
\newcommand{\NormalTok}[1]{#1}
\usepackage{longtable,booktabs}
\usepackage{graphicx,grffile}
\makeatletter
\def\maxwidth{\ifdim\Gin@nat@width>\linewidth\linewidth\else\Gin@nat@width\fi}
\def\maxheight{\ifdim\Gin@nat@height>\textheight\textheight\else\Gin@nat@height\fi}
\makeatother
% Scale images if necessary, so that they will not overflow the page
% margins by default, and it is still possible to overwrite the defaults
% using explicit options in \includegraphics[width, height, ...]{}
\setkeys{Gin}{width=\maxwidth,height=\maxheight,keepaspectratio}
\IfFileExists{parskip.sty}{%
\usepackage{parskip}
}{% else
\setlength{\parindent}{0pt}
\setlength{\parskip}{6pt plus 2pt minus 1pt}
}
\setlength{\emergencystretch}{3em}  % prevent overfull lines
\providecommand{\tightlist}{%
  \setlength{\itemsep}{0pt}\setlength{\parskip}{0pt}}
\setcounter{secnumdepth}{5}
% Redefines (sub)paragraphs to behave more like sections
\ifx\paragraph\undefined\else
\let\oldparagraph\paragraph
\renewcommand{\paragraph}[1]{\oldparagraph{#1}\mbox{}}
\fi
\ifx\subparagraph\undefined\else
\let\oldsubparagraph\subparagraph
\renewcommand{\subparagraph}[1]{\oldsubparagraph{#1}\mbox{}}
\fi

%%% Use protect on footnotes to avoid problems with footnotes in titles
\let\rmarkdownfootnote\footnote%
\def\footnote{\protect\rmarkdownfootnote}

%%% Change title format to be more compact
\usepackage{titling}

% Create subtitle command for use in maketitle
\newcommand{\subtitle}[1]{
  \posttitle{
    \begin{center}\large#1\end{center}
    }
}

\setlength{\droptitle}{-2em}

  \title{R语言模型部署实战}
    \pretitle{\vspace{\droptitle}\centering\huge}
  \posttitle{\par}
    \author{徐静}
    \preauthor{\centering\large\emph}
  \postauthor{\par}
      \predate{\centering\large\emph}
  \postdate{\par}
    \date{2018-08-06}

\usepackage{booktabs}
\usepackage{xeCJK}

\setCJKmainfont{宋体}

\setmainfont{Georgia}

\setromanfont{Georgia}

\setmonofont{Courier New}

\usepackage{amsthm}
\newtheorem{theorem}{Theorem}[chapter]
\newtheorem{lemma}{Lemma}[chapter]
\theoremstyle{definition}
\newtheorem{definition}{Definition}[chapter]
\newtheorem{corollary}{Corollary}[chapter]
\newtheorem{proposition}{Proposition}[chapter]
\theoremstyle{definition}
\newtheorem{example}{Example}[chapter]
\theoremstyle{definition}
\newtheorem{exercise}{Exercise}[chapter]
\theoremstyle{remark}
\newtheorem*{remark}{Remark}
\newtheorem*{solution}{Solution}
\begin{document}
\maketitle

{
\setcounter{tocdepth}{1}
\tableofcontents
}
\chapter*{序言}
\addcontentsline{toc}{chapter}{序言}

我们的模型不能只停留在线下的分析报告中,训练好的R模型如何应用到生产环境?目前针对于R语言的模型生产环境应用的方式有很多,比如用其他语言去调用,Java,Python等语言均可方便的调用R脚本;生成PMML文件,目前R中主流的一些R模型均支持PMML比如xgboost,lightGBM等,其他语言不需要调用R脚本只需调用统一的PMML文件就可以;还有就是Web端的部署,比如可以做成REST
API供其他语言调用,或直接做成web应用供其他用户访问,本书主要针对于R语言模型的Web端的部署。过程中,我们会先后介绍httpuv,opencpu,plumber,
jug,fiery,Rserve,RestRserve,等一些和模型线上化部署相关的R包(当然shiny也可以,但他不是我们本书的重点),最后会介绍mailR和Rweixin两个R和邮件与微信通信的R包,用于线上化部署的监测。当然会有其他的线上化部署方式。

欢迎进入R模型线上化部署的海洋!

\chapter*{关于我}
\addcontentsline{toc}{chapter}{关于我}

\textbf{徐静:}

硕士研究生,
目前的研究兴趣主要包括:数理统计,统计机器学习,深度学习,网络爬虫,前端可视化,R语言和Python语言的超级粉丝,多个R包和Python模块的作者,现在正逐步向Java迁移。

Graduate students,the current research interests include: mathematical
statistics, statistical machine learning, deep learning, web crawler,
front-end visualization. He is a super fan of R and Python, and the
author of several R packages and Python modules, and now gradually
migrating to Java.

\chapter{httpuv}\label{httpuv}

在httpuv的官网中,有这么一段描述:

\begin{quote}
Allows R code to listen for and interact with HTTP and WebSocket
clients, so you can serve web traffic directly out of your R process.
Implementation is based on libuv and http-parser.
\end{quote}

\begin{quote}
This is a low-level library that provides little more than network I/O
and implementations of the HTTP and WebSocket protocols. For an easy way
to create web applications, try Shiny instead.
\end{quote}

我们可以通过httpuv搭建一个访问R模型的web API,但可能这不是最好的。

本部分我们首先介绍官方提供的一些方法,然后解析官方提供的演示Demo,从而达到熟练使用httpuv的目的。

\section{方法介绍}

下面我们解析一下httpuv官方提供的一些调用方法,并演示一些调用方法对应的实例

1.使用URI编码/解码以与Web浏览器相同的方式对字符串进行编码/解码。

\begin{Shaded}
\begin{Highlighting}[]
\KeywordTok{encodeURI}\NormalTok{(value)}
\KeywordTok{encodeURIComponent}\NormalTok{(value)}
\KeywordTok{decodeURI}\NormalTok{(value)}
\KeywordTok{decodeURIComponent}\NormalTok{(value)}
\end{Highlighting}
\end{Shaded}

参数列表

\begin{itemize}
\tightlist
\item
  value 用于编码和解码的字符向量,UTF-8字符编码
\end{itemize}

\begin{Shaded}
\begin{Highlighting}[]
\KeywordTok{library}\NormalTok{(httpuv)}
\NormalTok{value <-}\StringTok{ "https://baidu.com/中国;?/"}
\KeywordTok{encodeURI}\NormalTok{(value)}
\end{Highlighting}
\end{Shaded}

\begin{verbatim}
## [1] "https://baidu.com/%D6%D0%B9%FA;?/"
\end{verbatim}

\begin{Shaded}
\begin{Highlighting}[]
\KeywordTok{encodeURIComponent}\NormalTok{(value)}
\end{Highlighting}
\end{Shaded}

\begin{verbatim}
## [1] "https%3A%2F%2Fbaidu.com%2F%D6%D0%B9%FA%3B%3F%2F"
\end{verbatim}

\begin{Shaded}
\begin{Highlighting}[]
\KeywordTok{decodeURI}\NormalTok{(value)}
\end{Highlighting}
\end{Shaded}

\begin{verbatim}
## [1] "https://baidu.com/中国;?/"
\end{verbatim}

\begin{Shaded}
\begin{Highlighting}[]
\KeywordTok{decodeURIComponent}\NormalTok{(value)}
\end{Highlighting}
\end{Shaded}

\begin{verbatim}
## [1] "https://baidu.com/中国;?/"
\end{verbatim}

注意: encodeURI 与 encodeURIComponent是不一样的因为前者不对特殊字符:
;,/?:@\&=+\$等进行encode

2.中断httpuv运行的环路

\begin{Shaded}
\begin{Highlighting}[]
\KeywordTok{interrupt}\NormalTok{()}
\end{Highlighting}
\end{Shaded}

\begin{enumerate}
\def\labelenumi{\arabic{enumi}.}
\setcounter{enumi}{2}
\tightlist
\item
  检查ip地址的类型是ipv4还是ipv6
\end{enumerate}

\begin{Shaded}
\begin{Highlighting}[]
\KeywordTok{ipFamily}\NormalTok{(ip)}
\end{Highlighting}
\end{Shaded}

参数列表

\begin{itemize}
\item
  ip 一个代表IP地址的字符串
\item
  返回值的意义:如果是IPv4返回4,如果是IPv6返回6,如果不是IP地址返回-1
\end{itemize}

\begin{Shaded}
\begin{Highlighting}[]
\KeywordTok{ipFamily}\NormalTok{(}\StringTok{"127.0.0.1"}\NormalTok{) }\CommentTok{# 4}
\end{Highlighting}
\end{Shaded}

\begin{verbatim}
## [1] 4
\end{verbatim}

\begin{Shaded}
\begin{Highlighting}[]
\KeywordTok{ipFamily}\NormalTok{(}\StringTok{"500.0.0.500"}\NormalTok{) }\CommentTok{# -1}
\end{Highlighting}
\end{Shaded}

\begin{verbatim}
## [1] -1
\end{verbatim}

\begin{Shaded}
\begin{Highlighting}[]
\KeywordTok{ipFamily}\NormalTok{(}\StringTok{"500.0.0.500"}\NormalTok{) }\CommentTok{# -1}
\end{Highlighting}
\end{Shaded}

\begin{verbatim}
## [1] -1
\end{verbatim}

\begin{Shaded}
\begin{Highlighting}[]
\KeywordTok{ipFamily}\NormalTok{(}\StringTok{"::"}\NormalTok{) }\CommentTok{# 6}
\end{Highlighting}
\end{Shaded}

\begin{verbatim}
## [1] 6
\end{verbatim}

\begin{Shaded}
\begin{Highlighting}[]
\KeywordTok{ipFamily}\NormalTok{(}\StringTok{"::1"}\NormalTok{) }\CommentTok{# 6}
\end{Highlighting}
\end{Shaded}

\begin{verbatim}
## [1] 6
\end{verbatim}

\begin{Shaded}
\begin{Highlighting}[]
\KeywordTok{ipFamily}\NormalTok{(}\StringTok{"fe80::1ff:fe23:4567:890a"}\NormalTok{) }\CommentTok{# 6}
\end{Highlighting}
\end{Shaded}

\begin{verbatim}
## [1] 6
\end{verbatim}

3.将原始向量转换为BASE64编码字符串

\begin{Shaded}
\begin{Highlighting}[]
\KeywordTok{rawToBase64}\NormalTok{(x)}
\end{Highlighting}
\end{Shaded}

参数列表

\begin{itemize}
\tightlist
\item
  x 原始向量
\end{itemize}

\begin{Shaded}
\begin{Highlighting}[]
\KeywordTok{set.seed}\NormalTok{(}\DecValTok{100}\NormalTok{)}
\NormalTok{result <-}\StringTok{ }\KeywordTok{rawToBase64}\NormalTok{(}\KeywordTok{as.raw}\NormalTok{(}\KeywordTok{runif}\NormalTok{(}\DecValTok{19}\NormalTok{, }\DataTypeTok{min=}\DecValTok{0}\NormalTok{, }\DataTypeTok{max=}\DecValTok{256}\NormalTok{)))}
\CommentTok{#stopifnot(identical(result, "TkGNDnd7z16LK5/hR2bDqzRbXA=="))}
\NormalTok{result}
\end{Highlighting}
\end{Shaded}

\begin{verbatim}
## [1] "TkGNDnd7z16LK5/hR2bDqzRbXA=="
\end{verbatim}

4.运行一个server

\begin{Shaded}
\begin{Highlighting}[]
\KeywordTok{runServer}\NormalTok{(host, port, app, }\DataTypeTok{interruptIntervalMs =} \OtherTok{NULL}\NormalTok{)}
\end{Highlighting}
\end{Shaded}

参数列表

\begin{itemize}
\item
  host IPv4地址, 或是``0.0.0.0''监听所有的IP
\item
  port 端口号
\item
  app 一个定义应用的函数集合
\item
  interruptIntervalMs 该参数不提倡使用,1.3.5版本后废除
\end{itemize}

\begin{Shaded}
\begin{Highlighting}[]
\NormalTok{app <-}\StringTok{ }\KeywordTok{list}\NormalTok{(}\DataTypeTok{call =} \ControlFlowTok{function}\NormalTok{(req)\{}
  \KeywordTok{list}\NormalTok{(}\DataTypeTok{status=}\NormalTok{200L,}
       \DataTypeTok{headers =} \KeywordTok{list}\NormalTok{(}
         \StringTok{'Content-Type'}\NormalTok{ =}\StringTok{ 'text/html'}
\NormalTok{       ),}
       \DataTypeTok{body =} \StringTok{"HelloWorld!"}\NormalTok{)}
\NormalTok{\})}

\KeywordTok{runServer}\NormalTok{(}\StringTok{"0.0.0.0"}\NormalTok{, }\DecValTok{5000}\NormalTok{,app)}
\end{Highlighting}
\end{Shaded}

5.过程请求

处理HTTP请求和WebSocket消息。 如果R的调用堆栈上没有任何东西,如果R是
在命令提示符下闲置,不必调用此函数,因为请求将
自动处理。但是,如果R正在执行代码,则请求将不被处理。
要么调用栈是空的,要么调用这个函数(或者,调用run\_now())。

\begin{Shaded}
\begin{Highlighting}[]
\KeywordTok{service}\NormalTok{(}\DataTypeTok{timeoutMs =} \KeywordTok{ifelse}\NormalTok{(}\KeywordTok{interactive}\NormalTok{(), }\DecValTok{100}\NormalTok{, }\DecValTok{1000}\NormalTok{))}
\end{Highlighting}
\end{Shaded}

参数列表

\begin{itemize}
\tightlist
\item
  timeoutMs 返回之前运行的毫秒数。
\end{itemize}

6.创建HTTP/WebSocket后台服务器(弃用)

\begin{Shaded}
\begin{Highlighting}[]
\KeywordTok{startDaemonizedServer}\NormalTok{(host, port, app)}
\end{Highlighting}
\end{Shaded}

7.创建HTTP/WebSocket服务器

\begin{Shaded}
\begin{Highlighting}[]
\KeywordTok{startServer}\NormalTok{(host, port, app)}
\KeywordTok{startPipeServer}\NormalTok{(name, mask, app)}
\end{Highlighting}
\end{Shaded}

参数列表

\begin{itemize}
\item
  host ip地址
\item
  port 端口号
\item
  app 一个定义应用的函数集
\end{itemize}

\begin{Shaded}
\begin{Highlighting}[]

\NormalTok{app <-}\StringTok{ }\KeywordTok{list}\NormalTok{(}
  \DataTypeTok{call =} \ControlFlowTok{function}\NormalTok{(req) \{}
    \KeywordTok{list}\NormalTok{(}
      \DataTypeTok{status =}\NormalTok{ 200L,}
      \DataTypeTok{headers =} \KeywordTok{list}\NormalTok{(}
        \StringTok{'Content-Type'}\NormalTok{ =}\StringTok{ 'text/html'}
\NormalTok{        ),}
      \DataTypeTok{body =} \StringTok{"Hello world!"}
\NormalTok{    )}
\NormalTok{    \}}
\NormalTok{  )}
\NormalTok{handle <-}\StringTok{ }\KeywordTok{startServer}\NormalTok{(}\StringTok{"0.0.0.0"}\NormalTok{, }\DecValTok{5000}\NormalTok{,app)}

\CommentTok{# 此服务器的句柄,可以传递给StestServer以关闭服务器。}
\KeywordTok{stopServer}\NormalTok{(handle)}
\end{Highlighting}
\end{Shaded}

8.停止所有应用

\begin{Shaded}
\begin{Highlighting}[]
\KeywordTok{stopAllServers}\NormalTok{()}
\end{Highlighting}
\end{Shaded}

9.在UNIX环境中停止运行的后台服务器(弃用)

\begin{Shaded}
\begin{Highlighting}[]
\KeywordTok{stopDaemonizedServer}\NormalTok{(handle)}
\end{Highlighting}
\end{Shaded}

10.停止一个服务

\begin{Shaded}
\begin{Highlighting}[]
\KeywordTok{stopServer}\NormalTok{(handle)}
\end{Highlighting}
\end{Shaded}

\section{例子演示}

\begin{enumerate}
\def\labelenumi{\arabic{enumi}.}
\tightlist
\item
  json-server
\end{enumerate}

\begin{Shaded}
\begin{Highlighting}[]

\CommentTok{# Connect to this using websockets on port 9454}
\CommentTok{# Client sends to server in the format of \{"data":[1,2,3]\}}
\CommentTok{# The websocket server returns the standard deviation of the sent array}
\KeywordTok{library}\NormalTok{(jsonlite)}
\KeywordTok{library}\NormalTok{(httpuv)}

\CommentTok{# Server}
\NormalTok{app <-}\StringTok{ }\KeywordTok{list}\NormalTok{(}
  \DataTypeTok{onWSOpen =} \ControlFlowTok{function}\NormalTok{(ws) \{}
\NormalTok{    ws}\OperatorTok{$}\KeywordTok{onMessage}\NormalTok{(}\ControlFlowTok{function}\NormalTok{(binary, message) \{}
      \CommentTok{# Decodes message from client}
\NormalTok{      message <-}\StringTok{ }\KeywordTok{fromJSON}\NormalTok{(message)}
      \CommentTok{# Sends message to client}
\NormalTok{      ws}\OperatorTok{$}\KeywordTok{send}\NormalTok{(}
        \CommentTok{# JSON encode the message}
        \KeywordTok{toJSON}\NormalTok{(}
          \CommentTok{# Returns standard deviation for message}
          \KeywordTok{sd}\NormalTok{(message}\OperatorTok{$}\NormalTok{data)}
\NormalTok{        )}
\NormalTok{      )}
\NormalTok{    \})}
\NormalTok{  \}}
\NormalTok{)}
\KeywordTok{runServer}\NormalTok{(}\StringTok{"0.0.0.0"}\NormalTok{, }\DecValTok{9454}\NormalTok{, app, }\DecValTok{250}\NormalTok{)}
\end{Highlighting}
\end{Shaded}

2.echo

\begin{Shaded}
\begin{Highlighting}[]
\KeywordTok{library}\NormalTok{(httpuv)}

\NormalTok{app <-}\StringTok{ }\KeywordTok{list}\NormalTok{(}
  \DataTypeTok{call =} \ControlFlowTok{function}\NormalTok{(req) \{}
\NormalTok{    wsUrl =}\StringTok{ }\KeywordTok{paste}\NormalTok{(}\DataTypeTok{sep=}\StringTok{''}\NormalTok{,}
                  \StringTok{'"'}\NormalTok{,}
                  \StringTok{"ws://"}\NormalTok{,}
                  \KeywordTok{ifelse}\NormalTok{(}\KeywordTok{is.null}\NormalTok{(req}\OperatorTok{$}\NormalTok{HTTP_HOST), req}\OperatorTok{$}\NormalTok{SERVER_NAME, req}\OperatorTok{$}\NormalTok{HTTP_HOST),}
                  \StringTok{'"'}\NormalTok{)}
    
    \KeywordTok{list}\NormalTok{(}
      \DataTypeTok{status =}\NormalTok{ 200L,}
      \DataTypeTok{headers =} \KeywordTok{list}\NormalTok{(}
        \StringTok{'Content-Type'}\NormalTok{ =}\StringTok{ 'text/html'}
\NormalTok{      ),}
      \DataTypeTok{body =} \KeywordTok{paste}\NormalTok{(}
        \DataTypeTok{sep =} \StringTok{"}\CharTok{\textbackslash{}r\textbackslash{}n}\StringTok{"}\NormalTok{,}
        \StringTok{"<!DOCTYPE html>"}\NormalTok{,}
        \StringTok{"<html>"}\NormalTok{,}
        \StringTok{"<head>"}\NormalTok{,}
        \StringTok{'<style type="text/css">'}\NormalTok{,}
        \StringTok{'body \{ font-family: Helvetica; \}'}\NormalTok{,}
        \StringTok{'pre \{ margin: 0 \}'}\NormalTok{,}
        \StringTok{'</style>'}\NormalTok{,}
        \StringTok{"<script>"}\NormalTok{,}
        \KeywordTok{sprintf}\NormalTok{(}\StringTok{"var ws = new WebSocket(%s);"}\NormalTok{, wsUrl),}
        \StringTok{"ws.onmessage = function(msg) \{"}\NormalTok{,}
        \StringTok{'  var msgDiv = document.createElement("pre");'}\NormalTok{,}
        \StringTok{'  msgDiv.innerHTML = msg.data.replace(/&/g, "&amp;").replace(/}\CharTok{\textbackslash{}\textbackslash{}}\StringTok{</g, "&lt;");'}\NormalTok{,}
        \StringTok{'  document.getElementById("output").appendChild(msgDiv);'}\NormalTok{,}
        \StringTok{"\}"}\NormalTok{,}
        \StringTok{"function sendInput() \{"}\NormalTok{,}
        \StringTok{"  var input = document.getElementById('input');"}\NormalTok{,}
        \StringTok{"  ws.send(input.value);"}\NormalTok{,}
        \StringTok{"  input.value = '';"}\NormalTok{,}
        \StringTok{"\}"}\NormalTok{,}
        \StringTok{"</script>"}\NormalTok{,}
        \StringTok{"</head>"}\NormalTok{,}
        \StringTok{"<body>"}\NormalTok{,}
        \StringTok{'<h3>Send Message</h3>'}\NormalTok{,}
        \StringTok{'<form action="" onsubmit="sendInput(); return false">'}\NormalTok{,}
        \StringTok{'<input type="text" id="input"/>'}\NormalTok{,}
        \StringTok{'<h3>Received</h3>'}\NormalTok{,}
        \StringTok{'<div id="output"/>'}\NormalTok{,}
        \StringTok{'</form>'}\NormalTok{,}
        \StringTok{"</body>"}\NormalTok{,}
        \StringTok{"</html>"}
\NormalTok{      )}
\NormalTok{    )}
\NormalTok{  \},}
  \DataTypeTok{onWSOpen =} \ControlFlowTok{function}\NormalTok{(ws) \{}
\NormalTok{    ws}\OperatorTok{$}\KeywordTok{onMessage}\NormalTok{(}\ControlFlowTok{function}\NormalTok{(binary, message) \{}
\NormalTok{      ws}\OperatorTok{$}\KeywordTok{send}\NormalTok{(message)}
\NormalTok{    \})}
\NormalTok{  \}}
\NormalTok{)}

\KeywordTok{browseURL}\NormalTok{(}\StringTok{"http://localhost:9454/"}\NormalTok{)}
\KeywordTok{runServer}\NormalTok{(}\StringTok{"0.0.0.0"}\NormalTok{, }\DecValTok{9454}\NormalTok{, app, }\DecValTok{250}\NormalTok{)}

\end{Highlighting}
\end{Shaded}

3.deamon-echo

\begin{Shaded}
\begin{Highlighting}[]
\KeywordTok{library}\NormalTok{(httpuv)}

\NormalTok{.lastMessage <-}\StringTok{ }\OtherTok{NULL}

\NormalTok{app <-}\StringTok{ }\KeywordTok{list}\NormalTok{(}
  \DataTypeTok{call =} \ControlFlowTok{function}\NormalTok{(req) \{}
\NormalTok{    wsUrl =}\StringTok{ }\KeywordTok{paste}\NormalTok{(}\DataTypeTok{sep=}\StringTok{''}\NormalTok{,}
                  \StringTok{'"'}\NormalTok{,}
                  \StringTok{"ws://"}\NormalTok{,}
                  \KeywordTok{ifelse}\NormalTok{(}\KeywordTok{is.null}\NormalTok{(req}\OperatorTok{$}\NormalTok{HTTP_HOST), req}\OperatorTok{$}\NormalTok{SERVER_NAME, req}\OperatorTok{$}\NormalTok{HTTP_HOST),}
                  \StringTok{'"'}\NormalTok{)}
    
    \KeywordTok{list}\NormalTok{(}
      \DataTypeTok{status =}\NormalTok{ 200L,}
      \DataTypeTok{headers =} \KeywordTok{list}\NormalTok{(}
        \StringTok{'Content-Type'}\NormalTok{ =}\StringTok{ 'text/html'}
\NormalTok{      ),}
      \DataTypeTok{body =} \KeywordTok{paste}\NormalTok{(}
        \DataTypeTok{sep =} \StringTok{"}\CharTok{\textbackslash{}r\textbackslash{}n}\StringTok{"}\NormalTok{,}
        \StringTok{"<!DOCTYPE html>"}\NormalTok{,}
        \StringTok{"<html>"}\NormalTok{,}
        \StringTok{"<head>"}\NormalTok{,}
        \StringTok{'<style type="text/css">'}\NormalTok{,}
        \StringTok{'body \{ font-family: Helvetica; \}'}\NormalTok{,}
        \StringTok{'pre \{ margin: 0 \}'}\NormalTok{,}
        \StringTok{'</style>'}\NormalTok{,}
        \StringTok{"<script>"}\NormalTok{,}
        \KeywordTok{sprintf}\NormalTok{(}\StringTok{"var ws = new WebSocket(%s);"}\NormalTok{, wsUrl),}
        \StringTok{"ws.onmessage = function(msg) \{"}\NormalTok{,}
        \StringTok{'  var msgDiv = document.createElement("pre");'}\NormalTok{,}
        \StringTok{'  msgDiv.innerHTML = msg.data.replace(/&/g, "&amp;").replace(/}\CharTok{\textbackslash{}\textbackslash{}}\StringTok{</g, "&lt;");'}\NormalTok{,}
        \StringTok{'  document.getElementById("output").appendChild(msgDiv);'}\NormalTok{,}
        \StringTok{"\}"}\NormalTok{,}
        \StringTok{"function sendInput() \{"}\NormalTok{,}
        \StringTok{"  var input = document.getElementById('input');"}\NormalTok{,}
        \StringTok{"  ws.send(input.value);"}\NormalTok{,}
        \StringTok{"  input.value = '';"}\NormalTok{,}
        \StringTok{"\}"}\NormalTok{,}
        \StringTok{"</script>"}\NormalTok{,}
        \StringTok{"</head>"}\NormalTok{,}
        \StringTok{"<body>"}\NormalTok{,}
        \StringTok{'<h3>Send Message</h3>'}\NormalTok{,}
        \StringTok{'<form action="" onsubmit="sendInput(); return false">'}\NormalTok{,}
        \StringTok{'<input type="text" id="input"/>'}\NormalTok{,}
        \StringTok{'<h3>Received</h3>'}\NormalTok{,}
        \StringTok{'<div id="output"/>'}\NormalTok{,}
        \StringTok{'</form>'}\NormalTok{,}
        \StringTok{"</body>"}\NormalTok{,}
        \StringTok{"</html>"}
\NormalTok{      )}
\NormalTok{    )}
\NormalTok{  \},}
  \DataTypeTok{onWSOpen =} \ControlFlowTok{function}\NormalTok{(ws) \{}
\NormalTok{    ws}\OperatorTok{$}\KeywordTok{onMessage}\NormalTok{(}\ControlFlowTok{function}\NormalTok{(binary, message) \{}
\NormalTok{      .lastMessage <<-}\StringTok{ }\NormalTok{message}
\NormalTok{      ws}\OperatorTok{$}\KeywordTok{send}\NormalTok{(message)}
\NormalTok{    \})}
\NormalTok{  \}}
\NormalTok{)}

\NormalTok{server <-}\StringTok{ }\KeywordTok{startDaemonizedServer}\NormalTok{(}\StringTok{"0.0.0.0"}\NormalTok{, }\DecValTok{9454}\NormalTok{, app)}

\CommentTok{# check the value of .lastMessage after echoing to check it is being updated}

\CommentTok{# call this after done}
\CommentTok{#stopDaemonizedServer(server)}

\end{Highlighting}
\end{Shaded}

\begin{enumerate}
\def\labelenumi{\arabic{enumi}.}
\setcounter{enumi}{3}
\item
\end{enumerate}

\begin{Shaded}
\begin{Highlighting}[]
\KeywordTok{library}\NormalTok{(httpuv)}
\NormalTok{app =}\StringTok{ }\KeywordTok{list}\NormalTok{(}\DataTypeTok{call =} \ControlFlowTok{function}\NormalTok{(req)\{}
  \CommentTok{# 获取POST的参数}
\NormalTok{  postdata =}\StringTok{ }\NormalTok{req}\OperatorTok{$}\NormalTok{rook.input}\OperatorTok{$}\KeywordTok{read_lines}\NormalTok{()}
\NormalTok{  qs =}\StringTok{ }\NormalTok{httr}\OperatorTok{:::}\KeywordTok{parse_query}\NormalTok{(}\KeywordTok{gsub}\NormalTok{(}\StringTok{"^}\CharTok{\textbackslash{}\textbackslash{}}\StringTok{?"}\NormalTok{, }\StringTok{""}\NormalTok{, postdata))}
\NormalTok{  dat =}\StringTok{ }\NormalTok{jsonlite}\OperatorTok{::}\KeywordTok{fromJSON}\NormalTok{(qs}\OperatorTok{$}\NormalTok{jsonDat)}
  \KeywordTok{print}\NormalTok{(dat)}
  \CommentTok{# 计算返回结果}
\NormalTok{  r =}\StringTok{ }\FloatTok{0.3} \OperatorTok{+}\StringTok{ }\FloatTok{0.1} \OperatorTok{*}\StringTok{ }\NormalTok{dat}\OperatorTok{$}\NormalTok{v1 }\OperatorTok{-}\StringTok{ }\FloatTok{0.2} \OperatorTok{*}\StringTok{ }\NormalTok{dat}\OperatorTok{$}\NormalTok{v2 }\OperatorTok{+}\StringTok{ }\FloatTok{0.1} \OperatorTok{*}\StringTok{ }\NormalTok{dat}\OperatorTok{$}\NormalTok{v3}
\NormalTok{  output =}\StringTok{ }\NormalTok{jsonlite}\OperatorTok{::}\KeywordTok{toJSON}\NormalTok{(}\KeywordTok{list}\NormalTok{(}\DataTypeTok{message =} \StringTok{'suceess'}\NormalTok{, }\DataTypeTok{result =}\NormalTok{ r), }\DataTypeTok{auto_unbox =}\NormalTok{ T)}
\NormalTok{  res =}\StringTok{ }\KeywordTok{list}\NormalTok{(}\DataTypeTok{status =}\NormalTok{ 200L, }\DataTypeTok{headers =} \KeywordTok{list}\NormalTok{(}\StringTok{'Content-Type'}\NormalTok{ =}\StringTok{ 'application/json'}\NormalTok{), }\DataTypeTok{body =}\NormalTok{ output)}
                                           \KeywordTok{return}\NormalTok{(res)}
\NormalTok{\})}
                                           \CommentTok{# 启动服务}
\NormalTok{                                           server =}\StringTok{ }\KeywordTok{startServer}\NormalTok{(}\StringTok{"0.0.0.0"}\NormalTok{, 1124L, }\DataTypeTok{app =}\NormalTok{ app)}
                                           \ControlFlowTok{while}\NormalTok{(}\OtherTok{TRUE}\NormalTok{) \{}
                                           \KeywordTok{service}\NormalTok{()}
                                           \KeywordTok{Sys.sleep}\NormalTok{(}\FloatTok{0.001}\NormalTok{)}
\NormalTok{                                           \}}
                                           \CommentTok{# stopServer(server)}

\end{Highlighting}
\end{Shaded}

\begin{Shaded}
\begin{Highlighting}[]

\NormalTok{RCurl}\OperatorTok{::}\KeywordTok{postForm}\NormalTok{(}\StringTok{'127.0.0.1:1124'}\NormalTok{,}
\DataTypeTok{style =} \StringTok{'post'}\NormalTok{,}
\DataTypeTok{.params =} \KeywordTok{list}\NormalTok{(}\DataTypeTok{jsonDat =} \StringTok{'\{"v1":1,"v2":2,"v3":3\}'}\NormalTok{)}
\NormalTok{)}
\end{Highlighting}
\end{Shaded}

httpuv是相对比较底层的包,熟练使用需要掌握前端知识,并且需要用到RCurl,httr相关爬虫包的一些知识去处理。本人不推荐这种方式进行模型的部署。

\chapter{opencpu}\label{opencpu}

\chapter{plumber}\label{plumber}

\chapter{jug}\label{jug}

\chapter{fiery}\label{fiery}

\chapter{Rserve}\label{rserve}

\chapter{RestRserve}\label{restrserve}

\chapter{mailR}\label{mailr}

mailR是一个比较小的包,主要解决的问题是R与邮件发送的问题,该包就一个方法:send.mail()
方法调用方式为:

\begin{Shaded}
\begin{Highlighting}[]
\KeywordTok{send.mail}\NormalTok{(from, to, }\DataTypeTok{subject =} \StringTok{""}\NormalTok{, }\DataTypeTok{body =} \StringTok{""}\NormalTok{, }\DataTypeTok{encoding =} \StringTok{"iso-8859-1"}\NormalTok{,}
\DataTypeTok{html =} \OtherTok{FALSE}\NormalTok{, }\DataTypeTok{inline =} \OtherTok{FALSE}\NormalTok{, }\DataTypeTok{smtp =} \KeywordTok{list}\NormalTok{(), }\DataTypeTok{authenticate =} \OtherTok{FALSE}\NormalTok{,}
\DataTypeTok{send =} \OtherTok{TRUE}\NormalTok{, }\DataTypeTok{attach.files =} \OtherTok{NULL}\NormalTok{, }\DataTypeTok{debug =} \OtherTok{FALSE}\NormalTok{, ...)}
\end{Highlighting}
\end{Shaded}

参数列表:

\begin{itemize}
\item
  from 有效的发送者的邮箱
\item
  to 目标接收的邮箱
\item
  subject 邮箱主题
\item
  body 邮件体
\item
  encoding 邮件内容字符编码 支持包括 iso-8859-1 (default), utf-8,
  us-ascii, and koi8-r
\item
  html bool值,是否把邮箱体解析成html
\item
  inline 布尔值,HTML文件中的图像是否应该嵌入内联。
\item
  smtp lsit类型,链接邮箱的smtp
\item
  authenticate 一个布尔变量,用于指示是否需要授权连接到
  SMTP服务器。如果设置为true,请参阅SMTP参数所需参数的详细信息。
  发送一个布尔值,指示电子邮件是否应该在函数的末尾发送。
  (默认行为)。如果设置为false,函数将电子邮件对象返回给父 环境。
\item
  attach.files 链接到文件的文件系统中路径的字符向量或\emph{有效}
  URL到附加到电子邮件(详见更多信息附加URL)
\item
  debug bool值,是否查看debug的真实细节
\item
  \ldots{} Optional arguments to be passed related to file attachments.
  See details for more
\end{itemize}

Example1:

\begin{Shaded}
\begin{Highlighting}[]
\NormalTok{mailR}\OperatorTok{::}\KeywordTok{send.mail}\NormalTok{(}
  \DataTypeTok{from =} \StringTok{'sender@tuandai.com'}\NormalTok{, }\CommentTok{# 发送人}
  \DataTypeTok{to =} \StringTok{'sendee@tuandai.com'}\NormalTok{, }\CommentTok{# 接收人}
  \DataTypeTok{cc =} \StringTok{'carboncopy@tuandai.com'}\NormalTok{, }\CommentTok{# 抄送人}
  \DataTypeTok{subject =} \StringTok{'邮件标题'}\NormalTok{,}
  \DataTypeTok{body =} \KeywordTok{as.character}\NormalTok{(}
    \StringTok{'<div style = "color:red">邮件正文,可以为HTML格式</div>'}
\NormalTok{  ),}
  \DataTypeTok{attach.files =} \OtherTok{NULL}\NormalTok{, }\CommentTok{# 附件的路径}
  \DataTypeTok{encoding =} \StringTok{"utf-8"}\NormalTok{,}
  \DataTypeTok{smtp =} \KeywordTok{list}\NormalTok{(}
    \DataTypeTok{host.name =} \StringTok{'smtp.exmail.qq.com'}\NormalTok{, }\CommentTok{# 邮件服务器IP地址}
    \DataTypeTok{port =} \DecValTok{465}\NormalTok{, }\CommentTok{# 邮件服务器端口}
    \DataTypeTok{user.name =} \StringTok{'senderName'}\NormalTok{, }\CommentTok{# 发送人名称}
    \DataTypeTok{passwd =} \StringTok{'yourpassword'}\NormalTok{, }\CommentTok{# 密码}
    \DataTypeTok{ssl =}\NormalTok{ T),}
  \DataTypeTok{html =}\NormalTok{ T, }\DataTypeTok{inline =}\NormalTok{ T, }\DataTypeTok{authenticate =}\NormalTok{ T, }\DataTypeTok{send =}\NormalTok{ T, }\DataTypeTok{debug =}\NormalTok{ F}
\NormalTok{)}
\end{Highlighting}
\end{Shaded}

Example2:

\begin{Shaded}
\begin{Highlighting}[]
\KeywordTok{send.mail}\NormalTok{(}\DataTypeTok{from =} \StringTok{"sender@gmail.com"}\NormalTok{,}
          \DataTypeTok{to =} \KeywordTok{c}\NormalTok{(}\StringTok{"Recipient 1 <recipient1@gmail.com>"}\NormalTok{, }\StringTok{"recipient2@gmail.com"}\NormalTok{),}
          \DataTypeTok{cc =} \KeywordTok{c}\NormalTok{(}\StringTok{"CC Recipient <cc.recipient@gmail.com>"}\NormalTok{),}
          \DataTypeTok{bcc =} \KeywordTok{c}\NormalTok{(}\StringTok{"BCC Recipient <bcc.recipient@gmail.com>"}\NormalTok{),}
          \DataTypeTok{subject=}\StringTok{"Subject of the email"}\NormalTok{,}
          \DataTypeTok{body =} \StringTok{"Body of the email"}\NormalTok{,}
          \DataTypeTok{smtp =} \KeywordTok{list}\NormalTok{(}\DataTypeTok{host.name =} \StringTok{"aspmx.l.google.com"}\NormalTok{, }\DataTypeTok{port =} \DecValTok{25}\NormalTok{),}
          \DataTypeTok{authenticate =} \OtherTok{FALSE}\NormalTok{,}
          \DataTypeTok{send =} \OtherTok{TRUE}\NormalTok{)}
\end{Highlighting}
\end{Shaded}

Example3:

\begin{Shaded}
\begin{Highlighting}[]
\KeywordTok{send.mail}\NormalTok{(}\DataTypeTok{from =} \StringTok{"sender@gmail.com"}\NormalTok{,}
          \DataTypeTok{to =} \KeywordTok{c}\NormalTok{(}\StringTok{"recipient1@gmail.com"}\NormalTok{, }\StringTok{"recipient2@gmail.com"}\NormalTok{),}
          \DataTypeTok{subject =} \StringTok{"Subject of the email"}\NormalTok{,}
          \DataTypeTok{body =} \StringTok{"Body of the email"}\NormalTok{,}
          \DataTypeTok{smtp =} \KeywordTok{list}\NormalTok{(}\DataTypeTok{host.name =} \StringTok{"smtp.gmail.com"}\NormalTok{, }\DataTypeTok{port =} \DecValTok{465}\NormalTok{, }\DataTypeTok{user.name =} \StringTok{"gmail_username"}\NormalTok{, }\DataTypeTok{passwd =} \StringTok{"password"}\NormalTok{, }\DataTypeTok{ssl =} \OtherTok{TRUE}\NormalTok{),}
          \DataTypeTok{authenticate =} \OtherTok{TRUE}\NormalTok{,}
          \DataTypeTok{send =} \OtherTok{TRUE}\NormalTok{)}
\end{Highlighting}
\end{Shaded}

Example4:

\begin{Shaded}
\begin{Highlighting}[]

\NormalTok{email <-}\StringTok{ }\KeywordTok{send.mail}\NormalTok{(}\DataTypeTok{from =} \StringTok{"Sender Name <sender@gmail.com>"}\NormalTok{,}
                   \DataTypeTok{to =} \StringTok{"recipient@gmail.com"}\NormalTok{,}
                   \DataTypeTok{subject =} \StringTok{"A quote from Gandhi"}\NormalTok{,}
                   \DataTypeTok{body =} \StringTok{"In Hindi :  थोडा सा अभ्यास बहुत सारे उपदेशों से बेहतर है।}
\StringTok{                   English translation: An ounce of practice is worth more than tons of preaching."}\NormalTok{,}
                   \DataTypeTok{encoding =} \StringTok{"utf-8"}\NormalTok{,}
                   \DataTypeTok{smtp =} \KeywordTok{list}\NormalTok{(}\DataTypeTok{host.name =} \StringTok{"smtp.gmail.com"}\NormalTok{, }\DataTypeTok{port =} \DecValTok{465}\NormalTok{, }\DataTypeTok{user.name =} \StringTok{"gmail_username"}\NormalTok{, }\DataTypeTok{passwd =} \StringTok{"password"}\NormalTok{, }\DataTypeTok{ssl =}\NormalTok{ T),}
               \DataTypeTok{authenticate =} \OtherTok{TRUE}\NormalTok{,}
                   \DataTypeTok{send =} \OtherTok{TRUE}\NormalTok{)}
\end{Highlighting}
\end{Shaded}

Example5:

\begin{Shaded}
\begin{Highlighting}[]

\KeywordTok{send.mail}\NormalTok{(}\DataTypeTok{from =} \StringTok{"sender@gmail.com"}\NormalTok{,}
          \DataTypeTok{to =} \KeywordTok{c}\NormalTok{(}\StringTok{"recipient1@gmail.com"}\NormalTok{, }\StringTok{"recipient2@gmail.com"}\NormalTok{),}
          \DataTypeTok{subject =} \StringTok{"Subject of the email"}\NormalTok{,}
          \DataTypeTok{body =} \StringTok{"Body of the email"}\NormalTok{,}
          \DataTypeTok{smtp =} \KeywordTok{list}\NormalTok{(}\DataTypeTok{host.name =} \StringTok{"smtp.gmail.com"}\NormalTok{, }\DataTypeTok{port =} \DecValTok{465}\NormalTok{, }\DataTypeTok{user.name =} \StringTok{"gmail_username"}\NormalTok{, }\DataTypeTok{passwd =} \StringTok{"password"}\NormalTok{, }\DataTypeTok{ssl =} \OtherTok{TRUE}\NormalTok{),}
          \DataTypeTok{authenticate =} \OtherTok{TRUE}\NormalTok{,}
          \DataTypeTok{send =} \OtherTok{TRUE}\NormalTok{,}
          \DataTypeTok{attach.files =} \KeywordTok{c}\NormalTok{(}\StringTok{"./download.log"}\NormalTok{, }\StringTok{"upload.log"}\NormalTok{),}
          \DataTypeTok{file.names =} \KeywordTok{c}\NormalTok{(}\StringTok{"Download log"}\NormalTok{, }\StringTok{"Upload log"}\NormalTok{), }\CommentTok{# optional parameter}
          \DataTypeTok{file.descriptions =} \KeywordTok{c}\NormalTok{(}\StringTok{"Description for download log"}\NormalTok{, }\StringTok{"Description for upload log"}\NormalTok{))}
\end{Highlighting}
\end{Shaded}

Example6:

\begin{Shaded}
\begin{Highlighting}[]
\KeywordTok{send.mail}\NormalTok{(}\DataTypeTok{from =} \StringTok{"sender@gmail.com"}\NormalTok{,}
          \DataTypeTok{to =} \KeywordTok{c}\NormalTok{(}\StringTok{"recipient1@gmail.com"}\NormalTok{, }\StringTok{"recipient2@gmail.com"}\NormalTok{),}
          \DataTypeTok{subject =} \StringTok{"Subject of the email"}\NormalTok{,}
          \DataTypeTok{body =} \StringTok{"<html>The apache logo - <img src=}\CharTok{\textbackslash{}"}\StringTok{http://www.apache.org/images/asf_logo_wide.gif}\CharTok{\textbackslash{}"}\StringTok{></html>"}\NormalTok{, }\CommentTok{# can also point to local file (see next example)}
          \DataTypeTok{html =} \OtherTok{TRUE}\NormalTok{,}
          \DataTypeTok{smtp =} \KeywordTok{list}\NormalTok{(}\DataTypeTok{host.name =} \StringTok{"smtp.gmail.com"}\NormalTok{, }\DataTypeTok{port =} \DecValTok{465}\NormalTok{, }\DataTypeTok{user.name =} \StringTok{"gmail_username"}\NormalTok{, }\DataTypeTok{passwd =} \StringTok{"password"}\NormalTok{, }\DataTypeTok{ssl =} \OtherTok{TRUE}\NormalTok{),}
          \DataTypeTok{authenticate =} \OtherTok{TRUE}\NormalTok{,}
          \DataTypeTok{send =} \OtherTok{TRUE}\NormalTok{)}
\end{Highlighting}
\end{Shaded}

Example7:

\begin{Shaded}
\begin{Highlighting}[]
\KeywordTok{send.mail}\NormalTok{(}\DataTypeTok{from =} \StringTok{"sender@gmail.com"}\NormalTok{,}
          \DataTypeTok{to =} \KeywordTok{c}\NormalTok{(}\StringTok{"recipient1@gmail.com"}\NormalTok{, }\StringTok{"recipient2@gmail.com"}\NormalTok{),}
          \DataTypeTok{subject =} \StringTok{"Subject of the email"}\NormalTok{,}
          \DataTypeTok{body =} \StringTok{"path.to.local.html.file"}\NormalTok{,}
          \DataTypeTok{html =} \OtherTok{TRUE}\NormalTok{,}
          \DataTypeTok{inline =} \OtherTok{TRUE}\NormalTok{,}
          \DataTypeTok{smtp =} \KeywordTok{list}\NormalTok{(}\DataTypeTok{host.name =} \StringTok{"smtp.gmail.com"}\NormalTok{, }\DataTypeTok{port =} \DecValTok{465}\NormalTok{, }\DataTypeTok{user.name =} \StringTok{"gmail_username"}\NormalTok{, }\DataTypeTok{passwd =} \StringTok{"password"}\NormalTok{, }\DataTypeTok{ssl =} \OtherTok{TRUE}\NormalTok{),}
          \DataTypeTok{authenticate =} \OtherTok{TRUE}\NormalTok{,}
          \DataTypeTok{send =} \OtherTok{TRUE}\NormalTok{)}
\end{Highlighting}
\end{Shaded}

\chapter{Rweixin}\label{rweixin}

\chapter{参考文献}\label{reference}

\bibliography{book.bib,packages.bib}


\end{document}
